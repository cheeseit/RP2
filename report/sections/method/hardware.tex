To create a baseline, experiments were also done on the DAS-4 and OpenNebula.
\subsubsection{Das-4}
\label{hw:das4}
The Distributed ASCI Supercomputer 4 (DAS-4) is a six-cluster wide-area distributed system designed by the Advanced School for Computing and Imagining (ASCI)\cite{das-4}. The six-clusters are the following:
\begin{itemize}
\item Vrije Univeristeit (VU)
\item Leiden Univeristeit (LU)
\item Univeristeit van Amsterdam (UvA) 
\item Technische Universiteit Delft (TUD)
\item Univeristeit van Amsterdam - The MultimediaN (UVA-MN) 
\item Astronomy institute Netherlands Institute for Radio Astronomy (ASTRON)
\end{itemize}
All computations have been done on the clusters from the VU and LU. The LU cluster was not actively used at that time. The VU cluster has the most nodes available of the all the clusters and has an OpenNebula cluster installed. The specifications of the clusters can be found in table \ref{tab:das-clusters}. 

Each of the nodes in the clusters has a dual quad-core processor with a speed of 2.4 GHz. All the nodes are also connected with InfiniBand\cite{infiniband} to improve the inter-node communication. The nodes on the cluster have CentOS release 6.6 (Final) installed.
\begin{table}[!h]
	\begin{center}
\begin{tabular}{|l|l|l|}
\hline
Cluster & Number of nodes  & Memory(GB) \\ \hline
VU 		& 74 (41) for all purposed	 & 24			\\ \hline
LU		& 16 & 48 \\ \hline
\end{tabular}
\end{center}
\caption{The specifications of the clusters used in this project.}
\label{tab:das-clusters}
\end{table}

\subsubsection{OpenNebula}
\label{hw:opennebula}
To be able to compare the results from the public cloud to another cloud solution, OpenNebula was used. 
``OpenNebula  provides the most simple but feature-rich and flexible solution for the comprehensive management of virtualized data centers to enable private, public and hybrid IaaS clouds. OpenNebula interoperability makes cloud an evolution by leveraging existing IT assets, protecting your investments, and avoiding vendor lock-in.'' \cite{opennebula}. The OpenNebula(ON) software installed on the VU cluster is version 3.8. The ON cluster on the DAS-4 consists of eight nodes. Table \ref{tab:specs-opennebula} shows the specifications of the ON nodes. ON uses templates to define resources to give virtual machines.  The template needed to know how to instantiate virtual machines using a given image and makes the resource provisioning flexible. The created virtual machines used the following two templates seen in Appendix[Reference].
%TODO think of what to more to write about the templates.
\begin{table} [!h]
	\begin{center}
	\begin{tabular}{|l|l|l|}
		\hline
		CPU & Number of cores & RAM  \\ \hline
		Intel(R) Xeon(R) CPU E5-2620 0 2.00 GHz & 24 & 66 GB\\ \hline
	\end{tabular}
	\caption{The specifications of the OpenNebula host nodes.}
	\label{tab:specs-opennebula}
	\end{center}
\end{table}

\subsubsection{Amazon EC2 instance types}
\label{hw:Amazon}
The Amazon Elastic Compute Cloud(Amazon EC2) has many different types of instances, each optimized for specific purposes\cite{amazon-instances}. The following types are available:
\begin{description}
\item[T - Burstable Performance Instances] provide burst of CPU with a low baseline. This means that it cannot maintain a constant high CPU load.
\item[M - Balanced] Well balanced in terms of memory, CPU and network resources.
\item[C -Computation Optimized] Used for computationally intensive tasks.
\item[R - Memory Optimized] Used for memory intensive tasks.
\item[G - GPU] Has a GPU.
\item[I - High I/O] Has large SSDs for fast I/O.
\item[D - Dense Storage] Has large HDD for storing a lot data.
\end{description}
From these types, two types are interesting for this project, namely: C and R. They were selected because we want to determine whether using a compute- or memory-optimized instance will make a difference for the performance of the benchmark.
The R and C types both have five different sizes of instances and different generations of hardware. Generation three has been chosen for both the R and C type. Size ``large'' is the smallest size which can be used for this type. We chose this size and generation because these are the cheapest instances that fit the project\footnote{ All experiments have been run on the Ireland region}. The specifications c3.large and r3.large can be seen in table \ref{tab:specs-amazon}.

\begin{table}[!h]
\begin{center}
\begin{tabular}{|l|p{1.8cm}|p{1.5cm}|l|}
\hline
Type & Number of vCPUs & Memory (GB) & Processor \\ \hline
c3.large & 2 & 4 & Intel Xeon E5-2680 v2 (Ivy Bridge) Processors 2.80 GHz \\ \hline
r3.large & 2 & 16 & Intel Xeon E5-2670 v2 (Ivy Bridge) Processors 2.50 GHz \\ \hline
\end{tabular}
\end{center}
\caption{The specifications of the Amazon EC2 instances chosen for this project.}
\label{tab:specs-amazon}
\end{table}
