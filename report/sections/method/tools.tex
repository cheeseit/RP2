\subsubsection{\texttt{graph500\_mpi\_simple}}

\subsubsection{Message Passing Interface and OpenMP}
\label{mpiopenmp}
The reference code of the Graph 500 is created in C and uses MPI and OpenMP to parallelize the program.
``Message Passing Interface (MPI) is a standardized and portable message-passing system designed by a group of researchers from academia and industry to function on a wide variety of parallel computers.\cite{mpi}''


``The OpenMP ARB mission is to standardize directive-based multi-language high-level parallelism that is performant, productive and portable''\cite{openmp}. In the Graph500 reference implementation OpenMP is often used to parallelize loops over multiple cores. 


\subsubsection{Intel MPI Benchmark}
\label{tools-imb}
Because the Graph 500 is a communication intensive benchmark, we need to understand the inter-node communication performance. For this we use the Intel MPI benchmark(IMB)  
IMB is used to determine out how well MPI performs on a certain platform. This is a free benchmark that can be used to measure the MPI functions on specific setups. The benchmark consists of a few different tests which all test a different aspect of MPI. Of these tests ``PingPong'' is the one important to us. The ``PingPong'' is used to measure the start up and throughput of a single message sent between two processes\cite{img-userguide}.

 The program is compiled using the OpenMPI compiler, details of the compilation can be found in Appendix [reference compilation benchmark].

