Modern HPC applications are optimized to process 3D physics simulation, but are not well suited for analyzing graphs. The Graph500 \cite{murphy2010introducing} is an initiative to improve the state of algorithms, hardware architectures and software systems to better suit graph analysis/processing. Graph500 does this by having a top 500 list where different institutes can compete and find out who has the best graph processor.

To get on the list, a benchmark should be run. This benchmark needs to comply to some specifications which can be found on this website\cite{graph500-specs}, but beyond these specifications anything goes. This means that the benchmark can be optimized for the hardware it runs on. 

The benchmark consist of two kernels, the first being the graph construction, the second is the Breadth-First Search. Both of these kernels will be timed. The kernels are preceded by the creation of an edge-list using a Kronecker Generator\cite{leskovec2010kronecker}. The results of both kernels are validated and the performance information is output.
\\
The list, at the moment, consists mostly of super computers. The aim of this project try and put an entry on the list using the public cloud. The focus will be on finding a model to make a prediction of how many machines will be needed to get a certain performance.